%%% 2017年 03月 22日 星期三 10:00:24 CST

%% Run: pdflatex -shell-escape ./blog43.tex


\documentclass[a4paper]{article}

\usepackage[top=3cm,left=2cm,right=2cm,bottom=2cm]{geometry}
\usepackage[backend=bibtex]{biblatex}
\usepackage[pdftex]{hyperref}
\usepackage{xcolor}
\usepackage{minted}
\usepackage{CJKutf8}

\begin{document}

\begin{CJK}{UTF8}{gkai}
	\title{一个玩具---自制人脸识别}
	\author{李小丹}
	\date{2017-03-22}

	\maketitle
\end{CJK}

\begin{CJK}{UTF8}{gbsn}

	\section*{自己动手}
	\href{http://opencv.org/}{OpenCV}提供了一些人脸识别的工具,比如FaceRecognizer.
	但自制一个人脸识别器是件挺好玩的事情,我用kNN实现了一个。

	\section*{三步实现}
	\begin{itemize}
		\item 数据预处理
		\item train
		\item 检查预测的效果
	\end{itemize}

	\section*{数据来源}
	数据使用\href{http://www.cl.cam.ac.uk/research/dtg/attarchive/facedatabase.html}
	{AT\&T Facedatabase}。
	还需要生成一个\href{http://docs.opencv.org/3.1.0/da/d60/tutorial_face_main.html#tutorial_face_prepare}{label文件}。
	也可以cd到数据存放的路径,使用这样的shell命令来生成这个文件:
	\begin{minted}{shell}
j=0;
while [ $j -le 39 ]; do i=1;while [ $i -le 10 ];
do echo "/home/shuheng/文档/att_faces/s$((j+1))/$i.pgm;$j"; i=$((i+1));
done; j=$((j+1)); done > a.txt
	\end{minted}

	\section*{实现步骤}
	\begin{description}
		\item [读取数据] 我借用了\href{http://docs.opencv.org/3.1.0/da/d60/tutorial_face_main.html#tutorial_face_fisherfaces_use}
			{read\_csv}。它来自OpenCV tutorial中的一个例子。
		\item [数据预处理] 利用OpenCV中(cvtColor + convertTo + reshape)对图片
			进行一系列的处理,最后形成一个TrainData。
		\item [实例化kNN] {\verb|KNearest::create|}
		\item [模型训练]  {\verb|KNearest::train|}
		\item [预测]   {\verb|KNearest::predict|}
	\end{description}

	\section*{代码}
	代码存放在\href{https://github.com/leeshuheng/toy_knn_face_rec}{github}。

\clearpage\end{CJK}

\end{document}
